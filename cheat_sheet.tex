\documentclass[10pt,landscape]{article}
\usepackage{multicol}
\usepackage{calc}
\usepackage[landscape]{geometry}
\usepackage{amsmath,amsthm,amsfonts,amssymb,relsize}
\usepackage{color,graphicx,overpic}

% Turn off header and footer
\pagestyle{empty}
\geometry{top=.25in,left=.25in,right=.25in,bottom=.25in}

% Redefine section commands to use less space
\makeatletter
\renewcommand{\section}{\@startsection{section}{1}{\z@}{3ex}{2ex}
                       {\normalfont\large\bfseries\textit}}
\renewcommand{\subsection}{\@startsection{subsection}{2}{\z@}{2ex}{0.5ex}
                          {\normalfont\normalsize\bfseries}}
\renewcommand{\subsubsection}{\@startsection{subsubsection}{3}{\z@}{-1ex}{0.5ex}
                             {\normalfont\small\bfseries}}
\makeatother

% Don't print section numbers
\setcounter{secnumdepth}{0}

\setlength{\parindent}{0pt}
\setlength{\parskip}{0pt}

% Math character command.
\newcommand{\R}{\mathbb{R}}
\newcommand{\E}{\mathbb{E}}

\newcommand{\bb}{\mathbf{b}}
\newcommand{\bv}{\mathbf{v}}
\newcommand{\bx}{\mathbf{x}}
\newcommand{\by}{\mathbf{y}}

\newcommand{\bA}{\mathbf{A}}
\newcommand{\bI}{\mathbf{I}}
\newcommand{\bX}{\mathbf{X}}
\newcommand{\bY}{\mathbf{Y}}
\newcommand{\bZ}{\mathbf{Z}}

\newcommand{\bbeta}{\boldsymbol{\beta}}
\newcommand{\btheta}{\boldsymbol{\theta}}
\newcommand{\bmu}{\boldsymbol{\mu}}
\newcommand{\bxi}{\boldsymbol{\xi}}
\newcommand{\bSigma}{\boldsymbol{\Sigma}}
\newcommand{\bLambda}{\boldsymbol{\Lambda}}

\newcommand{\tp}[1]{#1^{\prime}}
\newcommand{\inv}[1]{#1^{-1}}
\newcommand{\norm}[1]{\|#1\|}
\newcommand{\gaussian}[2]{\mathcal{N}(#1, #2)}
\newcommand{\inR}[2]{#1 \in \R^{#2}}
\newcommand{\ruler}{\\\rule{\columnwidth}{0.25pt}\\}

% -----------------------------------------------------------------------
\begin{document}
\raggedright
\footnotesize
\begin{multicols*}{3}
% multicol parameters
% These lengths are set only within the two main columns
% \setlength{\columnseprule}{0.1pt}
\setlength{\premulticols}{1pt}
\setlength{\postmulticols}{1pt}
\setlength{\multicolsep}{1pt}
\setlength{\columnsep}{0pt}


\section{Probability}
\textbf{Bayes Theorem}:\\
$P(Y=\pm 1 | X) = \frac{P(X|Y=\pm 1)P(Y=\pm 1)}{P(X|Y= +1)P(Y= +1) + P(X|Y= -1)P(Y= -1)}$
% --------------------------------

\section{Perceptron}
$f(\bx) = \btheta \cdot \bx + \theta_0 = \sum_{i=1}^{d}\theta_ix_i + \theta_0$,
$
\hat{y} =
\begin{cases}
    1,  & \text{if } f(x)\geq 0\\
    -1, & \text{if } f(x) < 0
\end{cases}
$
\ruler
\textbf{Decision boundary}, a \textbf{hyperplane} in $\R^d$: $H = \{\inR{\bx}{d} : f(\bx) = 0\} = \{\inR{\bx}{d} : \btheta \cdot \bx + \theta_0 = 0\}$
\ruler
$\btheta$ is the \textbf{normal} of the hyperplane,\\
$\theta_0$ is the \textbf{offset} of the hyperplane from origin,\\
$-\frac{\theta_0}{\norm{\btheta}}$ is the \textbf{signed distance} from the origin to hyperplane.
\ruler
\textbf{Perceptron algorithm},\\
Input: $(\bx_1, y_1),\dots,(\bx_n, y_n) \in \R^{d} \times \{\pm 1\}$\\
while some $y_i \neq \text{sign}(\btheta \cdot \bx_i)$\\
\-\hspace{0.5cm} pick some misclassified $(\bx_i, y_i)$\\
\-\hspace{0.5cm} $\btheta \leftarrow \btheta + y_i\bx_i$
\ruler
Given a \textbf{linearly separable data}, perceptron algorithm will take no more than $\frac{R^2}{\gamma^2}$ updates to \textbf{converge},
where $R = \max_{i}{\norm{\bx_i}}$ is the radius of the data, $\gamma = \min_{i}{\frac{y_i(\btheta\cdot\bx_i)}{\norm{\btheta}}}$ is the margin.\\
Also, $\frac{\btheta\cdot\bx}{\norm{\btheta}}$ is the signed distance from H to $\bx$ in the direction $\btheta$.
\ruler
$\btheta = \sum_{i} \alpha_i y_i \bx_i$, thus any inner product space will work, this is a \textbf{kernel}.
\ruler
\textbf{Gradient descent} view of perceptron, minimize margin cost function
$J(\btheta) = \sum_{i}(-y_i(\btheta\cdot\bx_i))_{+}$ with $\btheta \leftarrow \btheta - \eta\nabla J(\btheta)$
% --------------------------------

\section{Support Vector Machine}
\textbf{Hard margin SVM},\\
$\min_{\btheta} \norm{\btheta}^2$, such that $y_i\btheta\cdot\bx_i \geq 1 (i = 1,\dots,n)$\\
\textbf{Soft margin SVM},\\
$\min_{\btheta} \norm{\btheta}^2 + C\sum_{i=1}^{n}(1-y_i\btheta\cdot\bx_i)_+$
\ruler
\textbf{Regularization and SVMs}:
Simulated data with many features $\phi(\bx)$;
C controls trade-off between margin $1/\norm{\btheta}$ and fit to data;
Large C: focus on fit to data (small margin is ok). More overfitting.
Small C: focus on large margin, less tendency to overfit.
Overfitting increases with: less data, more features.
\ruler
$\btheta = \sum_{j}\alpha_j y_j\bx_j$, $\alpha_j \neq 0$ only for support vectors.
\ruler
$K(\bx_i, \bx_j) = \phi(\bx_i)\cdot\phi(\bx_j)$, K is called a kernel.\\
Solve $\alpha_j$ to determine $\sum_j\alpha_j y_j\phi(\bx_j)$.\\
Compute the classifier for a test point $\bx$ via $\btheta\cdot\phi(\bx) = \sum_{j}\alpha_j y_j K(\bx_j, \bx)$
\ruler
degree-m polynomial kernel: $K_m(\bx, \tilde{\bx}) = (1+\bx\cdot\tilde{\bx})^m$\\
radial basis function kernel: $K_{rbf}(\bx, \tilde{\bx}) = \text{exp}(-\gamma\norm{\bx-\tilde{\bx}}^2)$
% --------------------------------

\section{Decision Theory}
\textbf{Loss function}: $l : \mathcal{Y} \times \mathcal{Y} \rightarrow \R$,
and $l(\hat{y}, y)$ is the cost of predicting $\hat{y}$ when the outcome is $y$.
\ruler
Assume $(\bX, \bY)$ are chosen i.i.d according to some probability distribution on $\mathcal{X}\times\mathcal{Y}$.
\textbf{Risk} is misclassification probability: $R(f) = \E l(f(\bX), \bY) = Pr(f(\bX) \neq \bY)$
\ruler
\textbf{Bayes Decision Rule} is
$
f^{*}(x) =
\begin{cases}
    1,  & \text{if } P(\bY=1|x) > P(\bY=-1|x) \\
    -1, & \text{otherwise. }
\end{cases}
$,\\
and the optimal risk (Bayes risk) $R^{*} = \inf_{f}R(f)=R(f^*)$
\ruler
\textbf{Excess risk} is for any $f: \mathcal{X} \rightarrow \{-1, +1\}$,\\
$R(f) - R^* = \E (1[f(x)\neq f^*(x)]|2P(\bY=+1|\bX) - 1|)$
\ruler
\textbf{Risk in Regression} is expected  squared error:
$R(f) = \E l(f(\bX), \bY) = \E \E [f(\bX) - \bY^2 | \bX]$
\ruler
\textbf{Bias-variance decomposition}:
$R(f) = \E[\underbrace{(f(\bX) - \E[\bY|\bX])^2}_{\text{bias}^2}] + \E[\underbrace{(\E[\bY|\bX]-\bY)^2}_{\text{variance}}]$\\
$R(f) = \E[(f(\bX) - f^*(\bX))^2] + \E[(f^*(\bX)-\bY)^2]$\\
$R(f) = \E[(f(\bX) - f^*(\bX))^2] + R(f*)$\\
$R(f) - R(f*) = \E[(f(\bX) - f^*(\bX))^2]$, $f^*(\bX) = \E[\bY|\bX]$\\
% --------------------------------

\section{Generative and Discriminative}
\textbf{Discriminative models}: $P(\bX, \bY) = P(\bX)P(\bY|\bX)$.\\
Estimate $P(\bY|\bX)$, then pretend out estimate $\hat{P}(\bY|\bX)$ is the actual $P(\bY|\bX)$ and
plug in bayes rule expression.
\ruler
\textbf{Generative model}: $P(\bX, \bY) = P(\bY)P(\bX|\bY)$.\\
Estimate $P(\bY)$ and $P(\bX|\bY)$, then use bayes theorem to calculate $P(\bY|\bX)$ and use discriminative model.
\ruler
\textbf{Gaussian} class conditional densities $P(\bX|Y=+1)$,$P(\bX|Y=-1)$ (with the same variance), the posterior probability is \textbf{logistic}:\\
$P(Y=+1|\bx) = \frac{1}{1+\text{exp}(-\bx\cdot\bbeta-\beta_0)}$,\\
$\bbeta = \inv{\Sigma}(\bmu_{1}-\bmu_{0})$, $\beta_0=\frac{\tp{\bmu_{0}}\inv{\Sigma}\bmu0-\bmu_{1}\inv{\Sigma}\bmu1}{2}+\log\frac{P(Y=1)}{P(Y=0)}$
% --------------------------------

\section{Estimation}
\textbf{Method of moments}:
Match moments of the distribution to momemnts measured in the data.
\ruler
\textbf{Maximum likelihood}:
Choose parameter so that the distribution it defines gives the obverved data the highest probability (likelihood).
\ruler
\textbf{Maximum log likelihood}:
Log of maximum likelihood, equilvalent to maximum likelihood since log is monotonically increase; it is useful
since it can change $\prod$ to $\sum$.
\ruler
\textbf{Penalized maximum likelihood}:
Add a penalty term in the maximum (log) likelihood equation; treat the penalty term as some imaginary data points
crafted for desired probability.
% \ruler
\textbf{Bayesian estimate}:
Treat parameter as a random variable, then update based on observed value (data).\\
Prior: $\pi(p) = 1$,\\
Posterior: $P(p|\bX_1 = 1) = P(\bX_1 = 1|p)\pi(p) / \int P(X_1 = 1|q)d\pi(q)$
% \\\vspace{10pt}
\ruler
\textbf{Maximum a posterior probability}: the mode of the posterior. If uniform prior, MAP is MLE; if not
uniform prior, MAP is Penalized MLE.
% --------------------------------

\section{Multivariate Normal Distribution}
$\inR{\bx}{d}: p(x) = \frac{1}{(2\pi)^{d/2}|\bSigma|^{1/2}}e^{(-\frac{1}{2}\tp{(\bx-\bmu)}\inv{\bSigma}(\bx-\bmu))}$
\ruler
\textbf{Covariance matrix}:
$\bSigma = \E(\bX - \bmu)\tp{(\bX - \bmu)}$\\
Symmetric: $\bSigma_{i,j} = \bSigma_{j_i}$\\
Non-negative diagonal entries: $\bSigma{i,i} \geq 0$\\
Positive semidefinite: $\forall \inR{\bv}{d}, \tp{\bv}\bSigma\bv \geq 0$
\ruler
\textbf{Super-level} sets of pdf:
$\bxi_r = \left\{ \inR{\bx}{d} : \tp{(\bx - \bmu)}\inv{\bSigma}(\bx-\bmu)\leq r^2 \right\}$.\\
$\text{Volume of } \bxi_r \propto \prod^{d}_{i=1}\sigma_i = \sqrt{|\bSigma|}$
\ruler
\textbf{Spectral Theorem} for \textbf{non-diagonal covariance}:\\
$U = [\bv_1, \bv_2, \dots, \bv_n], \bLambda = \text{diag}(\tp{[\lambda_1, \lambda_2, \dots, \lambda_n]})$\\
We can eigen decompose $\inv{\bSigma} = U\inv{\bLambda}\tp{U}$, this is like to change to a different eigen spaces, where covariances ($\bLambda$) diagonal axis-alianed.
\ruler
Assume independent,
$ \gaussian{\bmu_x}{\bSigma} + \gaussian{\bmu_y}{\bSigma_y} = \gaussian{\bmu_x+\bmu_y}{\bSigma_x+\bSigma_y}$
\ruler
Given a $d$-dimensaional Gaussian $\bX \sim \gaussian{\bmu}{\bSigma}$,\\
write $\bX = \left[ \begin{smallmatrix} \bY \\ \bZ \end{smallmatrix} \right]$,
$\bmu = \left[ \begin{smallmatrix} \bmu_\bY \\ \bmu_\bZ \end{smallmatrix} \right]$,
$\bSigma = \left[ \begin{smallmatrix} \bSigma_{\bY\bY} & \bSigma_{\bY\bZ} \\ \bSigma_{\bZ\bY} & \bSigma_{\bZ\bZ} \end{smallmatrix} \right]$,\\
where $\inR{\bY}{m}$, and $\inR{\bZ}{d-m}$. Then $\bY \sim \gaussian{\mu_\bY}{\bSigma_{\bY\bY}}$
\ruler
Given a $d$-dimensaional Gaussian $\bX \sim \gaussian{\bmu}{\bSigma}$,\\
matrix $\inR{\bA}{m\times d}$ and vector $\inR{\bb}{m}$, define $\bY = \bA\bX + \bb$.\\
Then $\bY \sim \gaussian{\bA\bmu + \bb}{\bA\bSigma\tp{\bA}}$
\ruler
Given a $d$-dimensaional Gaussian $\bX \sim \gaussian{\bmu}{\bSigma}$,\\
with $\bSigma$ positive definite,\\
$\bY = \bSigma^{-\frac{1}{2}}(\bX-\bmu) \sim \gaussian{\mathbf{0}, \bI}$
\ruler
\textbf{Gaussian maximum likelihood estimation}:\\
Sample mean: $\hat{\bmu} = \frac{1}{n}\sum_{i=1}^{n}\bx_i$;\\
Sample covariance: $\hat{\bSigma} = \frac{1}{n}\sum_{i=1}^{n}(\bx_i - \hat{\bmu})\tp{(\bx_i - \hat{\bmu})}$
% --------------------------------

\section{Linear Regression}
Given $\inR{\bX}{p}$, $Y \in \R$, consider linear(affine) prediction rules,\\
$F_{\text{lin}} := \{ \bx \mapsto \tp{\bx}\bbeta + \beta_0 : \inR{\bbeta}{p}, \beta_0 \in \R \}$
\subsection{Empirical risk minimization}
\textbf{Empirical risk} is the sample average of squared error:\\
$\hat{R}(f) = \hat{\E}_n \ell(f(\bX), Y) = \frac{1}{n}\sum_{i=1}{n}(f(\bX_i) - Y_i)^2$\\
Choose $\hat{f} := \text{arg } min_{f \in F_\text{lin}} \hat{\E}_n \ell(f(\bX), Y)$
\\\vspace{100pt}
% \ruler
Find $\hat{f} : \bx \mapsto \tp{\bx}\hat{\bbeta}$, such that\\
$
\hat{\bbeta} = \text{arg } min_{\inR{\bbeta}{p}}\sum_{i=1}^{n}(\tp{\bX_i}\bbeta - Y_i)^2
             = \text{arg } min_{\inR{\bbeta}{p}}\underbrace{\norm{\bX\bbeta - \by}^2}_\text{RSS}
$\\
where \textbf{design matrix} $\inR{\bX}{n \times p}$ and \textbf{response vector} $\inR{\by}{n}$.
\ruler
\textbf{Normal equations}:
$\tp{\bX}\bX\bbeta = \tp{\bX}\by \Rightarrow \hat{\bbeta} = \inv{(\tp{\bX}\bX)}\tp{\bX}\by$
\ruler
\textbf{Projection Theorem} also leads to normal equations:\\
$\inv{(\by-\hat{\by})}\bX = 0 \Leftrightarrow \tp{\bX}(\by-\bX\bbeta) = 0 \Leftrightarrow \tp{\bX}\by=\tp{\bX}\bX\bbeta$
\subsection{Linear model with additive Gaussian noise}
Model the conditional distribution of $Y$ given $\bX = \bx$ as:
$P(Y|\bX=\bx) = \gaussian{\tp{\bx}\bbeta}{\sigma^2}$\\
Equivalently: $Y = \tp{\bx}\bbeta + \epsilon$, where $\epsilon \sim \gaussian{\mathbf{0}}{\sigma^2}$
\ruler
\textbf{Maximum likelihood} is least square:\\
$
L(\bbeta) = \prod_{i=1}^{n}p(Y_i|\bX_i,\bbeta) \Leftrightarrow
\ell(\bbeta) = g(\sigma^2)-\frac{1}{2\sigma^2}\sum_{i=1}^{n}(y_i-\tp{\bX}\bbeta)^2
$
\ruler
Fix $\bX$. Provided $\E\by=\bX\bbeta$ and $\text{Cov}(\by)=\sigma^2\bI$
\ruler
\textbf{Bayesian analysis}:
Treat $\bbeta$ as a r.v. with prior distribution $\gaussian{\mathbf{0}}{\tau^2\bI}$,
then compute posterior distribution $P(\bbeta|\bX, Y)$.
\ruler
$P(\bbeta|\bX_1, Y_1,\dots,\bX_n,Y_n) \propto P(Y_1,\dots,Y_n|\bbeta,\bX_1,\dots,\bX_n)P(\bbeta)$\\
$P(\bbeta|\bX_1, Y_1,\dots,\bX_n,Y_n) \propto exp(-\frac{1}{2}(\sum_{i=1}^{n}\frac{(Y_i-\tp{\bX}_i\bbeta)^2}{\sigma^2}+\frac{1}{\tau^2}\norm{\bbeta}^2))$
% --------------------------------

\section{Linear Regression Regularization}
\textbf{Trading off bias and variance}: some increase in bias can give a big decrease in variance.
\ruler
\textbf{Subset selection} is like $L0$ regularization:
RSS decreases as the complexity increases because the best fit with a smaller subset is always possible with a larger subset.
\ruler
\textbf{Find a path through subset space}: using cross-validation and forward-stepwise selection or backward-stepwise selection (need $n > p$).
\ruler
\textbf{Ridge regression} is like $L2$ regularization:
$\hat{\bbeta} = \text{arg }\min_{\bbeta}(\sum_{i=1}^n(y_i-\tp{\bx}_i\bbeta)^2 + \lambda\sum_{j=1}{p}\beta_j^2)$\\
$\hat{\bbeta}^{\text{ridge}}=\inv{(\tp{\bX}\bX+\lambda\bI)}\tp{\bx}\by$
\ruler
\textbf{Lasso regression} is like $L1$ regularization:
$\hat{\bbeta} = \text{arg }\min_{\bbeta}(\sum_{i=1}^n(y_i-\tp{\bx}_i\bbeta)^2 + \lambda\sum_{j=1}{p}|\beta_j|)$
\ruler
While ridge regression leads to reduced, but non-zero values of the coefficients,
Lasso regression forces some coefficients to be zero.
\ruler
\textbf{Bayesian analysis}:
Ridge regression is equivalent to a MAP estimate with a gaussian prior.
Lasso regression is equivalent to a MAP estimate with a Laplace prior.
% --------------------------------

\section{Logistic Regression}
Model \textbf{log odds} ($\log p/(1-p)$) as an affine function of $\bx$.
\ruler
$P(Y=1|\bx)=\frac{1}{1+\text{exp}(\tp{\bbeta}\bx)}$
Given data $(\bX_1, Y_1),\dots,(\bX_n, Y_n) \in \R^{p} \times \{0,1\}$,
estimate $\bbeta$ with maximum likelihood.
\ruler
\textbf{Log likelihood}:\\
$\ell(\bbeta) = \sum_{i=1}^{n}y_i\log \mu_i(\bbeta)+(1-y_i)\log(1-\mu_i(\bbeta))$,\\
where $\mu_i(\bbeta) = P(Y=1|\bX=\bx_i,\bbeta)=\frac{1}{1+\text{exp}(-\tp{\bbeta}\bx_i)}$
\ruler
$\nabla_{\bbeta}\mu_i(\bbeta) = \mu_i(\bbeta)(1-\mu_i(\bbeta))\bx_i$\\
$\nabla_{\bbeta}\ell(\bbeta)=\sum_{i=1}^{n}(y_i-\mu_i(\bbeta))\bx_i = \tp{\bX}(\by-\bmu)$\\
$\nabla_{\bbeta}^2\ell(\bbeta)=\sum_{i=1}^{n}-\mu_i(\bbeta)(1-\mu_i(\bbeta))\bx_i\tp{\bx}_i = -\tp{\bX}\text{diag}(\bmu(1-\bmu))\bX$
$\hat{\bbeta}^{\text{ml}}$ solves: $\sum_{i=1}^{n}y_i\bx_i=\sum_{i=1}^{n}\mu_i{\bbeta}\bx_i$
\ruler
\textbf{Gradient ascent}:\\
$\bbeta^{(t+1)}=\bbeta^{(t)}+\eta\nabla_{\bbeta}\ell(\bbeta^{(t)})$ : $O(np)$/step\\
\textbf{Stochastic gradient ascent}:\\
$\bbeta^{(t+1)}=\bbeta^{(t)}+\eta(y_{i_{t}}-\mu_{i_{t}}(\bbeta^{(t)}))\bx_{i_{t}}$ : $O(p)$/step\\
\textbf{Newton-Raphson method}:\\
$\bbeta^{(t+1)}=\bbeta^{(t)}-\inv{[\nabla_{\bbeta}^2\ell(\bbeta^{(t)})]}\nabla_{\bbeta}\ell(\bbeta^{(t)})$\\

% \section{Linear Discriminant Analysis}
% \section{Convex Optimization}
\end{multicols*}
\end{document}
